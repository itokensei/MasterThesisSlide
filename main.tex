\documentclass{beamer} % スライド用のレイアウト
% \documentclass[dvipdfmx,cjk]{beamer} 
% \documentclass[dvipdfm,cjk]{beamer} %% オプションは環境や利用するプログラムに
% \documentclass[dvips,cjk]{beamer}   %% よって変える

% \AtBeginDvi{\special{pdf:tounicode 90ms-RKSJ-UCS2}} %% しおりが文字化けしないように
% \AtBeginDvi{\special{pdf:tounicode EUC-UCS2}}

\usepackage{luatexja} % Beamer単体では日本語を出力できないのでこれを使う
%\renewcommand{\kanjifamilydefault}{\gtdefault} % 漢字をゴシック体に指定

% \setbeamertemplate{navigation symbols}{} %% 右下のアイコンを消す

% \usetheme[block=fill]{metropolis} % usepackageじゃなくてusethemeだから注意
\usetheme{Madrid}             %% theme の選択
% \usetheme{Boadilla}           %% Beamer のディレクトリの中の
% \usetheme{CambridgeUS}        %% Madrid.sty を指定
% \usetheme{Antibes}            %% 色々と試してみるといいだろう
% \usetheme{Montpellier}        %% サンプルが beamer\doc に色々とある.
% \usetheme{Berkeley}
% \usetheme{Goettingen}
% \usetheme{Singapore}
% \usetheme{Szeged}

% \usecolortheme{rose}          %% colortheme を選ぶと色使いが変わる
% \usecolortheme{albatross}

% \useoutertheme{shadow}                 %% 箱に影をつける
% \usefonttheme{professionalfonts}       %% 数式の文字を通常の LaTeX と同じにする

% \setbeamercovered{transparent}         %% 消えている文字をうっすらと表示する

\setbeamertemplate{theorems}[numbered]  %% 定理に番号をつける
\newtheorem{thm}{Theorem}[section]
\newtheorem{proposition}[thm]{Proposition}
\theoremstyle{example}
\newtheorem{exam}[thm]{Example}
\newtheorem{remark}[thm]{Remark}
\newtheorem{question}[thm]{Question}
\newtheorem{prob}[thm]{Problem}

% \definecolor{katsuobushi}{HTML}{cd853f} % HTMLカラーコードで指定
% \setbeamercolor{frametitle}{bg=katsuobushi} % bgをfgにすると文字色が変わる
% {block title} % 全ブロックのタイトル.これのbgを指定するとブロックの本文もこれを薄くした色になる
% {block body} % 全ブロックの本文.
% {block title alarted} % alartblockのタイトル
% {block body alarted} % 同上本文
% {block title example} % exampleblockのタイトル
% {block body example} % 同上本文
% {normal text} % 本文.これのbgを指定するとスライドの背景色が変わる

\title{線型ホモトピー型理論を動機付けする\\
トポロジカル量子コンパイルの形式検証に向けて}
\author{322301073 伊藤 賢世}
\date{2025年1月30日}

\begin{document}
\begin{frame}
  \maketitle
\end{frame}

\section{イントロダクション}
\begin{frame}
  \frametitle{イントロダクション}
  \begin{block}{修論内容:サーベイ}
    Hisham Satiらの研究プロジェクト
    \begin{center}
      Quantum Certification
      via Linear Homotopy Types
    \end{center}
    を理解するための\alert{トポロジカル量子コンパイル}について
  \end{block}
  \begin{block}{研究概要}
    \begin{description}
      \item[主張] 信頼性のある量子計算を実現するための
        検証言語「QS」 と
        その基礎理論「線型ホモトピー型理論(LHoTT)」の提唱.
      \item[研究者] Hisham Sati,Urs Schreiber,David J. Meyer,\(\ldots\)
      \item[拠点] ニューヨーク大学アブダビ校
        Center for Quantum and
        Topological Systems (CQTS)
    \end{description}
  \end{block}
\end{frame}

\begin{frame}
  \frametitle{イントロダクション}
  \begin{block}{重要課題:量子コンピュータの実現}
    \begin{itemize}
      \item 量子コンピュータ\(=\)量子力学の原理を活用した計算機.
      \item いくつかの計算について,
            普及している(古典)コンピュータの性能を
            凌駕する量子コンピュータ上のアルゴリズムが見つかっている.
      \item MicrosoftやGoogleなども巨額の投資.
    \end{itemize}
    \begin{center}
      \(\longrightarrow\)
      量子コンピュータの実現に対する期待は計り知れない.
    \end{center}
  \end{block}

  \begin{block}{計算の信頼性における課題①:Decoherence}
    \begin{itemize}
      \item 量子コンピュータのデータ\(=\)量子系の状態
            (\alert{量子ビット}).
      \item 量子状態は外部環境の影響で壊れやすい
            (\alert{Decoherence問題}).
    \end{itemize}
    \begin{center}
      \(\longrightarrow\)
      エラーが多く,アルゴリズム通りに計算を実行させることは難しい.
    \end{center}
  \end{block}
\end{frame}

\begin{frame}
  \frametitle{イントロダクション}
  \begin{block}{アプローチ①:エラーの訂正}
    \begin{itemize}
      \item 古典コンピュータでも物理レベルではエラーが起きている.
            \begin{itemize}
              \item データに冗長性をもたせて,データを復元
                    (誤り訂正理論).
            \end{itemize}
      \item 量子ビットを増やして,
            量子コンピュータにも応用(量子誤り訂正).
    \end{itemize}
    \begin{center}
      \(\longrightarrow\)
      量子ビットを大規模に用意する問題に至る.
    \end{center}
  \end{block}

  \begin{block}{アプローチ②:エラーの低減}
    \begin{itemize}
      \item \alert{トポロジカル量子計算}\(=\)
            トポロジカルな情報で量子状態を構成.
      \item Decoherenceに影響されにくく,エラーが減る.
    \end{itemize}
    \begin{center}
      \(\longrightarrow\)
      訂正のための冗長な量子ビットは少なくて済む.
    \end{center}
  \end{block}

  \begin{block}{}
    ①と②を組み合わせて,decoherenceに対処.
  \end{block}
\end{frame}

\begin{frame}
  \frametitle{イントロダクション}
  \begin{exampleblock}{コンパイル}
    あるプログラミング言語で書かれたコードを
    別の言語のコードに変換すること.
    特に,実行可能なコードに変換すること.
    実行前に最適化が施されている.
    \begin{description}
      \item[Cコンパイラ] C言語→アセンブリ言語
      \item[アセンブラ] アセンブリ言語→機械語
    \end{description}
    古典コンピュータは最終的に機械語を実行している.
  \end{exampleblock}

  \begin{block}{量子コンパイル}
    \begin{description}
      \item[ソフトウェア側] 量子アルゴリズムを量子回路で記述する.
      \item[ハードウェア側] 実行可能な量子回路.
    \end{description}
  \end{block}
\end{frame}

\begin{frame}
  \frametitle{イントロダクション}
  \begin{block}{量子コンパイルのアルゴリズム}
    \begin{itemize}
      \item 任意の実行可能な量子回路を用意するのは現実的でない.
      \item 量子回路を構成する基本要素を準備して,固定する\\
            (有限個の\alert{基本ゲート}).
      \item 基本ゲートの組み合わせで,
            任意の量子アルゴリズムを正確に\\
            シミュレートすることはできない.
    \end{itemize}
    \begin{center}
      \(\longrightarrow\)
      任意の精度で\alert{近似的に}シュミレートする.\\
      (cf. Solovey・Kitaevアルゴリズム)
    \end{center}
  \end{block}

  \begin{block}{課題①の先:トポロジカル量子コンパイル}
    トポロジカル量子コンピュータの量子コンパイル.
  \end{block}
\end{frame}

\begin{frame}
  \frametitle{イントロダクション}
  \begin{alertblock}{計算の信頼性における課題②:コンパイルの正しさ}
    古典計算では,コンパイルの正しさが検証されている.
    \begin{itemize}
      \item コンパイルの仕様を検証言語で記述
      \item cf. Cコンパイラは検証言語「Coq」で検証されている.
    \end{itemize}
    \begin{center}
      \(\longrightarrow\)
      トポロジカル量子コンパイルの近似精度も検証したい.
    \end{center}
  \end{alertblock}

  \begin{block}{Satiらの提案:線型ホモトピー型理論(LHoTT)}
    \begin{itemize}
      \item 前提
    \end{itemize}
    \begin{center}
      \(\longrightarrow\)結論
    \end{center}
  \end{block}
\end{frame}

\begin{frame}
  \frametitle{イントロダクション}
  \begin{block}{タイトル}
    \begin{itemize}
      \item 前提
    \end{itemize}
    \begin{center}
      \(\longrightarrow\)結論
    \end{center}
  \end{block}
\end{frame}

\begin{frame}
  \frametitle{一枚目のスライド}
  普通に文中で\LaTeX コマンドが使用できます.(そらそう)
  \begin{block}{ブロックのタイトル}
    block環境を作成しbeginの第二引数にタイトルを渡すとこうなります.
  \end{block}
  \begin{alertblock}{アラート}
    alertblockを指定するとこんな風になります.
  \end{alertblock}
  \begin{exampleblock}{例}
    exampleblockを指定するとこうなります.
  \end{exampleblock}
\end{frame}

\begin{frame}
  \frametitle{横並べ}
  スライドで左右比較するときなどにこのようにします
  \begin{columns}
    \begin{column}{.4\linewidth} % 現在の環境の幅の0.4(=40%)の意
      \begin{block}{内容1}
        内容1内容1内容1内容1内容1内容1内容1内容1内容1内容1内容1内容1内容1内容1
      \end{block}
    \end{column}
    \begin{column}{.4\linewidth}
      \begin{block}{内容2}
        内容2内容2内容2内容2内容2内容2内容2内容2内容2内容2内容2内容2内容2内容2
      \end{block}
    \end{column}
  \end{columns}
  このとき各columnの幅は合計100\%にしてしまうとデザイン的にあんまよくないです
\end{frame}

\begin{frame}
  \frametitle{Overlay}
  \begin{itemize}
    \item 一個目
          \pause % このコマンドを打つごとに一時停止する
    \item 二個目
          \pause
    \item 三個目
          \pause
    \item 四個目
  \end{itemize}
\end{frame}

\begin{itemize}
  \item<1-> 一個目(一画面目以降表示)
  \item<2-> 二個目(二画面目以降表示)
  \item<3-> 三個目(三画面目以降表示)
  \item<1-2> 四個目(二画面目まで表示)
\end{itemize}

\begin{frame}
  \frametitle{Item以外のOverlay}
  \begin{block}{常に表示1}
  \end{block}
  \uncover<2->{二画面目以降のみ表示(uncover)}
  \begin{block}{常に表示2}
  \end{block}
  \only<2->{二画面目以降のみ表示(only)}
  \begin{block}{常に表示3}
  \end{block}
\end{frame}
\end{document}